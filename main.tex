%Dokumentklasse
\documentclass[a4paper,12pt,parskip=half,headsepline,DIV=12,numbers=noenddot]{scrartcl}

% Loading all packages and configuration present in the preamble file
\usepackage{preamble}

% ======= BEGIN DES DOKUMENTS =======
\begin{document}

% Definition Header Sections sollen in der Kopfzeile stehen; Kopfzeile mit Unterstrich
\automark[subsection]{section}
\KOMAoptions{headsepline=true}
%\ihead{Kopfzeile innen}
%\chead{Kopfzeile außen}
\ohead{\headmark}

% Definition footer \pagemark steht für Seitennummer
%\ifoot{Fußzeile innen}
%\cfoot{Fußzeile Mitte}
\ofoot{\pagemark}

% % ------------------ Titelseite ------------------
%%%%%%%%%%%%%%%%%%%%%%%%%%%%%%%%%%%%%%%%%%%%%%%%%%%%%%%%%%%%%%%%%%%
%                                                                 %                    
%                    Definition Deckblatt                         %
%                                                                 %
%%%%%%%%%%%%%%%%%%%%%%%%%%%%%%%%%%%%%%%%%%%%%%%%%%%%%%%%%%%%%%%%%%%

% true für Bachelorarbeit / false für Hausarbeit
\newbool{bachelorarbeit}
\setbool{bachelorarbeit}{false}

% Setze Fachbereich
\newcommand{\department}{Fachbereich II \\ Management und Informationssysteme}

% Setze Studiengang
\newcommand{\studyprogram}{Wirtschaftsinformatik B.Sc.}

% Setze Modulname (bachelorarbeit muss false sein)
\newcommand{\modulname}{Tolles Modul}

% Setze Dozent:in (bachelorarbeit muss false sein)
\newcommand{\auditor}{\textbf{Dozent:in:} \> Prof. Dr. Maxi Muster}

% Setze Gutachter:innen (bachelorarbeit muss true sein)
\newcommand{\firstauditor}{\textbf{Erstgutachter:} \> Prof. Dr. Maxi Mustermann}
\newcommand{\secondauditor}{\textbf{Zweitgutachterin:} \> Prof. Dr. Maxi Musterfrau}

% Setze Titel und Untertitel der Abreit 
\newcommand{\thetitle}{Titel}
\newcommand{\thesubtitle}{Untertitel}

% Setze Autor:in und MatNr.
\newcommand{\theauthor}{Robin Mustermensch}
\newcommand{\matriculationnumber}{00000}

% Abstand zwischen Name und MatNr. (siehe Deckblatt)
\newcommand{\myspace}{1.0cm}

% Muss in src/basic_structure/deckblatt.tex einkommentiert werden! 

\newcommand{\secondauthor}{\> Maxi Mustermensch \> MatNr. 00000\\}
\newcommand{\thirdauthor}{\> Maxi Mustermensch \> MatNr. 00000\\}
\newcommand{\fourthauthor}{\> Maxi Mustermensch \> MatNr. 00000\\}
\newcommand{\fifthauthor}{\> Maxi Mustermensch \> MatNr. 00000\\}

% PDF Metadaten
%\hypersetup{pdfinfo={
%Title={\thetitle},
%Author={\theauthor}
%}}

\hypersetup{pdfinfo={
			Title={\thetitle},
			Author={\theauthor}
		}}


%%%%%%%%%%%%%%%%%%%%%%%%%%%%%%%%%
%           Deckblatt           %
%%%%%%%%%%%%%%%%%%%%%%%%%%%%%%%%%
\setmainfont{TeX Gyre Adventor}
\thispagestyle{empty}
\begin{figure}[h!]
	\centering
	\includegraphics[width=0.6\textwidth]{./images/hs-logo.png}
\end{figure}
\begin{center}
	\large{\textbf{\department}}\\
	\large{\textbf{\studyprogram}}\\
	\vspace{1cm}
	\ifbool{bachelorarbeit}{
		\LARGE{\textbf{Bachelorarbeit}}\\
		\large{zur Erlangung des akademischen Grades \\ Bachelor of Science}\\
	}
	{
		\large{\textbf{Modul\\ \modulname}}\\
	}
	\vspace*{\fill}
	\line(1,0){450}\\
	\doublespacing
	\textbf{\Large{\thetitle}}\\
	\textbf{\large{\thesubtitle}}\\
	\line(1,0){450}\\
\end{center}
\vspace*{\fill}
\onehalfspacing
\small{
	\begin{flushleft}
		\begin{tabbing}
			\textbf{Vorgelegt von:} \hspace*{0.8cm}\= \theauthor \hspace*{\myspace}\= MatNr. \matriculationnumber \\

			%%%%%%%%%%%%%%%%%%%%%%%%%%%%%%%%%%%%%%%%%%%%%%
			%                                            %
			%  Hier weitere Autor:innen einkommentieren  %
			%   Müssen hauptdatei.tex definiert sein     %
			%	                                         %
			%%%%%%%%%%%%%%%%%%%%%%%%%%%%%%%%%%%%%%%%%%%%%%		
			%\secondauthor
			%\thirdauthor
			%\fourthauthor
			%\fifthauthor

			\textbf{Vorgelegt am:} \> \today\\
			\ifbool{bachelorarbeit}{
				\firstauditor\\
				\secondauditor\\
			}
			{
				\auditor\\
			}
		\end{tabbing}
	\end{flushleft}}
\setmainfont{TeX Gyre Termes}

\newpage

% Singlespacing (Zeilenabstand) (Default)
\singlespacing
\normalsize

% Abstract falls gewünscht
%\thispagestyle{empty}
%\input{abstract}
%\newpage

% Inhaltsverzeichnis anzeigen
\pagestyle{empty}
\tableofcontents
\newpage
\pagestyle{headings}

% Header für den Inhalt 
\KOMAoptions{headsepline=true}
\ohead{\headmark}

% % ------------------ Inhalt des Dokuments ------------------


\section{Einleitung}
Das hier ist die Einleitung
\section{Erster Teil}
\input{contents/2.erster.tex}
\section{Zweiter Teil}
\input{contents/3.zweiter.tex}
\section{Zusammenfassung}
\input{contents/4.zusammenfassung.tex}
\newpage


% % ------------------ Verzeichnisse ------------------
% Literaturverzeichnis anzeigen
\ohead{Literaturverzeichnis} % Korrektur für Header 
\phantomsection
\addcontentsline{toc}{section}{Literaturverzeichnis}
\renewcommand\refname{Literaturverzeichnis}
\printbibliography
\newpage

% Abbildungsverzeichnis anzeigen
\ohead{\headmark}
\listoffigures
\addcontentsline{toc}{section}{Abbildungsverzeichnis}
\newpage


% Tabellenverzeichnis anzeigen
\listoftables
\addcontentsline{toc}{section}{Tabellenverzeichnis}
\newpage

% Listingverzeichnis anzeigen
% \renewcommand{\listlistingname}{Listingverzeichnis}
% \listoflistings 
% \addcontentsline{toc}{section}{Listingverzeichnis}
% \newpage


% Abkürzungsverzeichnis anzeigen
%\ohead{Abkürzungsverzeichnis} % Korrektur für Header 
%\section*{Abkürzungsverzeichnis}
%\input{src/basic_structure/abkuerzungen.tex}
%\addcontentsline{toc}{section}{Abkürzungsverzeichnis}
%\newpage

% Selbstständigkeits Erklärung
\phantomsection
\addcontentsline{toc}{section}{Selbstständigkeitserklärung}

% Header für Erklärung
\ohead{Selbstständigkeitserklärung}

% Input Erklärung
\input{erklaerung.tex}

% ======= ENDE DES DOKUMENTS =======
% Leere Abschlussseite
%\newpage
%\thispagestyle{empty} % erzeugt Seite ohne Kopf- / Fusszeile
%\mbox{}

\end{document}